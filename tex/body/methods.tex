% !TEX TS-program = pdflatex
% !TEX encoding = UTF-8 Unicode
% !TEX root = ../main.tex
% !TEX spellcheck = en-US
% ****************************************************************************************
% File: methods.tex
% Author: Jakob Spindler
% Date: 2024-10-16
% ****************************************************************************************
\chapter{Methods}
\label{chapter:methods}

To lower the conducted emissions of the buck converter, the following steps are taken:
\begin{itemize}
    \item place a capacitor at the buck converters input between rails (also known as x-cap) to reduce the \GLS{acr:dm} emissions \autocite{HowGetBest}
    \item implement a \gls{acr:cmc} to reduce the \GLS{acr:cm} emissions \autocite{HowGetBest}
    \item place capacitors between the rails and ground (also known as y-caps) to reduce the \GLS{acr:cm} emissions \autocite{hegartyHowActiveEMI2023}
\end{itemize}



in practice, many other methods can be used to reduce the conducted emissions of a buck converter, such as the use of ferrite beads/cores or even active filtering to name a few \autocite{hegartyHowActiveEMI2023,damnjanovicComparisonDifferentStructures2006}. However, the above-mentioned methods are chosen for their simplicity and cost-effectiveness.


% EOF